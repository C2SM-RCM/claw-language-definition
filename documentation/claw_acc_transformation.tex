\subsection{Accelerator abstraction/helper transformation}

\subsubsection{Conditional primitive directives}
\lstset{emph={primitive_prefix,primitive_directives}, emphstyle=\itshape}
\begin{lstlisting}
!$claw primitive_prefix primitive_directives

! Example for OpenACC
!$claw acc routine seq

! Example for OpenMP
!$claw omp target enter data
\end{lstlisting}

Primitive compiler directive like OpenACC or OpenMP can be enabled conditionally
during the transformation. Examples on line 4 and 7 are conditional directives.
They will be transformed into standard OpenACC respectively OpenMP directives
only if the primitive directive language is specified for the transformation.

\subsubsection{Vector notation expansion}
\lstset{emph={name,group_id,clause}, emphstyle=\itshape}
\begin{lstlisting}
!$claw expand [induction(name [[,] name]...)] [fusion [group(group_id)]] [parallel] [acc([clause [[,] clause]...])]

  ! vector notation assignment(s)

[!$claw end expand]
\end{lstlisting}

Computations using the vector notation are not suitable to be parallelized with
language like OpenACC. The \textbf{expand} directive allows to
transform those notation with the corresponding do statements which are more
suitable for parallelization.

The goal of this directive is to pass from an vector notation assignment like
this:

\begin{lstlisting}
!$claw expand
A(1:n) = A(1+m:n+m) + B(1:n) * C(n+1:n+n)
\end{lstlisting}

To a do statement statement like this:

\begin{lstlisting}
DO i=1,n
  A(i) = A(i+m) + B(i) * C(n+i)
END DO
\end{lstlisting}

If the directive is used as a block directive, the assignments are wrapped in
a single do statement if their induction range match.

\begin{lstlisting}
!$claw expand
A(1:n) = A(1+m:n+m) + B(1:n) * C(n+1:n+n)
B(1:n) = B(1:n) * 0.5
!$claw end expand
\end{lstlisting}

To

\begin{lstlisting}
DO i=1,n
  A(i) = A(i+m) + B(i) * C(n+i)
  B(i) = B(i) * 0.5
END DO
\end{lstlisting}


\textbf{Options and details}
\begin{enumerate}
\item \textit{induction}: Allow to name the induction variable created for the
do statement.
\item \textit{fusion}: Allow the create do statement to be merged with other
loops. Options
are identical with the \textbf{loop-fusion} directive
\item \textit{parallel}: Wrap the extracted loop in a parallel region.
\item \textit{acc}: Define accelerator clauses that will be applied to the
generated loops.
\end{enumerate}

\textbf{Behavior with other directives}\\
Directives declared before the \textbf{expand} directive will be kept
in the generated code.

\textbf{Code example}\\
\label{expand1}
Example with vector notation following the pragma statement.

Original code
\begin{lstlisting}
SUBROUTINE vector_add
  INTEGER :: i = 10
  INTEGER, DIMENSION(0:9) :: vec1

  !$claw expand
  vec1(0:i) = vec1(0:i) + 10;
END SUBROUTINE vector_add
\end{lstlisting}

Transformed code
\begin{lstlisting}
SUBROUTINE vector_add
  INTEGER :: claw_i
  INTEGER :: i = 10
  INTEGER, DIMENSION(0:9) :: vec1

  ! CLAW transformation vector notation to do statement
  DO claw_i = 0, i
    vec1(claw_i) = vec1(claw_i) + 10;
  END DO
END SUBROUTINE vector_add
\end{lstlisting}

More code examples in the appendix. Example with the induction and acc clauses
(see example \ref{expand2}). Example with fusion clause (see example
\ref{expand3}). Example with 2-dimensional arrayx (see example
\ref{expand4}).
