
\section{Code examples}

\subsection{kcache 2}
\label{kcache2}
Assignment is created during the transformation. 

\textbf{Original code}
\begin{lstlisting}
DO j3 = ki3sc+1, ki3ec
  !$claw kcache data(array2)
  var1 = x * y - array2(j1, j3)
  var2 = z * array2(j1, j3)
END DO
\end{lstlisting}


\textbf{Transformed code}
\begin{lstlisting}
DO j3 = ki3sc+1, ki3ec
  array2_k = array2(j1,j3)
  var1 = x * y - array2_k
  var2 = z * array2_k
END DO
\end{lstlisting}

\subsection{kcache 3}
\label{kcache3}
Using \lstinline!init! and \lstinline!private! clause

\textbf{Original code}
\begin{lstlisting}
!$acc parallel
DO j3 = ki3sc+1, ki3ec
  var3 = array1(j1,j3-1)

  !$claw kcache data(array1) offset(0 -1) init private
  array1(j1,ki3sc) = x * y * z
  var1 = x * y - array1(j1, j3-1)
  var2 = z * array1(j1, j3-1)
END DO
!$acc end parallel
\end{lstlisting}


\textbf{Transformed code}
\begin{lstlisting}
!$acc parallel private(array1_k_m1)
DO j3 = ki3sc+1, ki3ec
  IF (j3 == ki3sc+1) THEN
    array1_k_m1 = array1(j1,j3-1)
  END IF
  var3 = array1_k_m1
  array1_k_m1 = x * y * z
  array1(j1,ki3sc) = array1_k_m1
  var1 = x * y - array1_k_m1
  var2 = z * array1_k_m1
END DO
!$acc end parallel
\end{lstlisting}

\subsection{loop-interchange 2}
\label{loop-interchange2}
Example with 3 levels of loop. 

\textbf{Original code}
\begin{lstlisting}
!$claw loop-interchange (k,i,j)
DO i=1, iend     ! loop at depth 0
  DO j=1, jend   ! loop at depth 1
    DO k=1, kend ! loop at depth 2
      ! loop body here
    END DO
  END DO
END DO
\end{lstlisting}

\textbf{Transformed code}
\begin{lstlisting}
! CLAW transformation (loop-interchange (k,i,j))
DO k=1, kend       ! loop at depth 2
  DO i=1, iend     ! loop at depth 0
    DO j=1, jend   ! loop at depth 1
      ! loop body here
    END DO
  END DO
END DO
\end{lstlisting}

\subsection{loop-interchange 3}
\label{loop-interchange3}
Example with OpenACC directives. 

\textbf{Original code}
\begin{lstlisting}
!$acc parallel
!$acc loop gang
!$claw loop-interchange
DO i=1, iend
  !$acc loop vector
  DO k=1, kend
    ! loop body here
  END DO
END DO
!$acc end parallel
\end{lstlisting}

\textbf{Transformed code}
\begin{lstlisting}
! CLAW transformation (loop-interchange i < -- > k)
!$acc parallel
!$acc loop gang
DO k=1, kend
  !$acc loop vector
  DO i=1, iend
    ! loop body here
  END DO
END DO
!$acc end parallel
\end{lstlisting}


\subsection{loop-fusion 2}
\label{loop-fusion2}
Example using the \textit{group} clause.

\textbf{Original code}
\begin{lstlisting}
DO k=1, iend
  !$claw loop-fusion group(g1)
  DO i=1, iend
    ! loop #1 body here
  END DO

  !$claw loop-fusion group(g1)
  DO i=1, iend
    ! loop #2 body here
  END DO

  !$claw loop-fusion group(g2)
  DO i=1, jend
    ! loop #3 body here
  END DO

  !$claw loop-fusion group(g2)
  DO i=1, jend
    ! loop #4 body here
  END DO
END DO
\end{lstlisting}


\textbf{Transformed code}
\begin{lstlisting}
DO k=1, iend
  ! CLAW transformation (loop-fusion group g1)
  DO i=1, iend
    ! loop #1 body here
    ! loop #2 body here
  END DO

  ! CLAW tranformation (loop-fusion group g2)
  DO i=1, jend
    ! loop #3 body here
    ! loop #4 body here
  END DO
END DO
\end{lstlisting}

\subsection{loop-fusion 3}
\label{loop-fusion3}
Example showing behavior with other directives (OpenACC). 

\textbf{Original code}
\begin{lstlisting}
!$acc parallel
!$acc loop gang
DO k=1, iend
  !$acc loop seq
  !$claw loop-fusion
  DO i=1, iend
    ! loop #1 body here
  END DO

  !$acc loop vector
  !$claw loop-fusion
  DO i=1, iend
    ! loop #2 body here
  END DO
END DO
!$acc end parallel
\end{lstlisting}

\textbf{Transformed code}
\begin{lstlisting}
!$acc parallel
!$acc loop gang
DO k=1, iend
  ! CLAW transformation (loop-fusion same block group)
  !$acc loop seq
  DO i=1, iend
    ! loop #1 body here
    ! loop #2 body here
  END DO
END DO
!$acc end parallel
\end{lstlisting}

\subsection{loop-fusion 4}
\label{loop-fusion4}
Example using the \textit{collapse} clause. 

\textbf{Original code}
\begin{lstlisting}
DO k=1, iend
  !$claw loop-fusion collapse(2)
  DO i=0, iend
    DO j=0, jend
      ! nested loop #1 body here
    END FO
  END DO

  !$claw loop-fusion collapse(2)
  DO i=0, iend
    DO j=0, jend
      ! loop #2 body here
    END DO
  END DO
END DO
\end{lstlisting}


\textbf{Transformed code}
\begin{lstlisting}
DO k=1, iend
  ! CLAW transformation (loop-fusion collapse(2))
  DO i=0, iend
    DO j=0, jend
      ! nested loop #1 body here
      ! nested loop #2 body here
    END DO
  END DO
END DO
\end{lstlisting}


\subsection{loop-extract 2}
\label{loop-extract2}
Example with \textit{fusion} clause. 

\textbf{Original code}
\begin{lstlisting}
PROGRAM main
  !$claw loop-extract(i=istart,iend) map(value1,value2:i) fusion group(g1)
  CALL xyz(value1, value2)

  !$claw loop-fusion group(g1)
  DO i = istart, iend
    ! some computation here
    print*,'Inside loop', i
  END DO
END PROGRAM main

SUBROUTINE xyz(value1, value2)
  REAL, INTENT (IN) :: value2(x:y), value2(x:y)

  DO i = istart, iend
    ! some computation with value1(i) and value2(i) here
  END DO
END SUBROUTINE xyz
\end{lstlisting}

\textbf{Transformed code}
\begin{lstlisting}
PROGRAM main
  !CLAW extracted loop
  DO i = istart, iend
    CALL xyz_claw(value1(i), value2(i))
    ! some computation here
    print*,'Inside loop', i
  END DO
END PROGRAM main

SUBROUTINE xyz(value1, value2)
  REAL, INTENT (IN) :: value2(x:y), value2(x:y)

  DO i = istart, iend
    ! some computation with value1(i) and value2(i) here
  END DO
END SUBROUTINE xyz

!CLAW extracted loop new subroutine
SUBROUTINE xyz_claw(value1, value2)
  REAL, INTENT (IN) :: value1, value2
  ! some computation with value1 and value2 here
END SUBROUTINE xyz_claw
\end{lstlisting}


\subsection{loop-extract 3}
\label{loop-extract3}
Example with more complicated mapping. 

\textbf{Original code}
\begin{lstlisting}
PROGRAM main
  !$claw loop-extract(i=istart,iend) map(value1,value2:i/j)
  CALL xyz(value1, value2)
END PROGRAM main

SUBROUTINE xyz(value1, value2, j)
  INTGER, INTENT(IN) :: j
  REAL  , INTENT(IN) :: value2(x:y), value2(x:y)

  DO i = istart, iend
    ! some computation with value1(j) and value2(j) here
  END DO
END SUBROUTINE xyz
\end{lstlisting}

\textbf{Transformed code}
\begin{lstlisting}
PROGRAM main
  !CLAW extracted loop
  DO i = istart, iend
    CALL xyz_claw(value1(i), value2(i))
  END DO
END PROGRAM main

SUBROUTINE xyz(value1, value2, j)
  INTGER, INTENT(IN) :: j
  REAL  , INTENT(IN) :: value2(x:y), value2(x:y)

  DO i = istart, iend
    ! some computation with value1(j) and value2(j) here
  END DO
END SUBROUTINE xyz

!CLAW extracted loop new subroutine
SUBROUTINE xyz_claw(value1, value2, j)
  INTGER, INTENT(IN) :: j
  REAL, INTENT (IN) :: value1, value2
  ! some computation with value1 and value2 here
END SUBROUTINE xyz_claw
\end{lstlisting}


\subsection{array-transform 2}
\label{array-transform2}
Example with \textit{induction} and \textit{acc} clauses

\textbf{Original code}
\begin{lstlisting}
SUBROUTINE vector_add
  INTEGER :: i = 10
  INTEGER, DIMENSION(0:9) :: vec1

  !$acc parallel
  !$claw array-transform induction(myinduc) acc(loop)
  vec1(0:i) = vec1(0:i) + 10;

  !$claw array-transform acc(loop)
  vec1(0:i) = vec1(0:i) + 1;
  !$acc end parallel
END SUBROUTINE vector_add
\end{lstlisting}


\textbf{Transformed code}
\begin{lstlisting}
SUBROUTINE vector_add
  INTEGER :: claw_i
  INTEGER :: myinduc
  INTEGER :: i = 10
  INTEGER, DIMENSION(0:9) :: vec1

  !$acc parallel

  ! CLAW transformation array notation vec1(0:i) to do loop
  !$acc loop
  DO myinduc = 0, i
    vec1(myinduc) = vec1(myinduc) + 10;
  END DO

  ! CLAW transformation array notation vec1(0:i) to do loop
  !$acc loop
  DO claw_i = 0, i
    vec1(claw_i) = vec1(claw_i) + 1;
  END DO
  
  !$acc end parallel
END SUBROUTINE vector_add
\end{lstlisting}

\subsection{array-transform 3}
\label{array-transform3}
Example with fusion clause
 
\textbf{Original code}
\begin{lstlisting}
SUBROUTINE vector_add
  INTEGER :: i = 10
  INTEGER, DIMENSION(0:9) :: vec1
  INTEGER, DIMENSION(0:9) :: vec2

  !$claw array-transform induction(claw_i) fusion
  vec1(0:i) = vec1(0:i) + 10;

  !$claw array-transform induction(claw_i) fusion
  vec2(0:i) = vec2(0:i) + 1;
END SUBROUTINE vector_add
\end{lstlisting}


\textbf{Transformed code}
\begin{lstlisting}
SUBROUTINE vector_add
  INTEGER :: claw_i
  INTEGER :: i = 10
  INTEGER, DIMENSION(0:9) :: vec1
  INTEGER, DIMENSION(0:9) :: vec2

  ! CLAW transformation array notation vec1(0:i) to do loop
  ! CLAW transformation array notation vec2(0:i) to do loop
  ! CLAW transformation fusion
  DO claw_i=0, i
    vec1(claw_i) = vec1(claw_i) + 10;
    vec2(claw_i) = vec2(claw_i) + 1;
  END DO
END SUBROUTINE vector_add
\end{lstlisting}


\subsection{array-transform 4}
\label{array-transform4}
Example with 2-dimensional arrays

\textbf{Original code}
\begin{lstlisting}
SUBROUTINE vector_add
INTEGER :: i = 10
INTEGER, DIMENSION(0:10,0:10) :: vec1
INTEGER, DIMENSION(0:10,0:10) :: vec2

vec1(0:i,0:i) = 0;
vec2(0:i,0:i) = 100;

!$claw array-transform
vec1(0:i,0:i) = vec2(0:i,0:i) + 10
END SUBROUTINE vector_add
\end{lstlisting}

\textbf{Transformed code}
\begin{lstlisting}
SUBROUTINE vector_add
INTEGER :: i = 10
INTEGER, DIMENSION(0:10,0:10) :: vec1
INTEGER, DIMENSION(0:10,0:10) :: vec2

vec1(0:i,0:i) = 0;
vec2(0:i,0:i) = 100;

DO claw_i = 0, i, 1
  DO claw_j = 0, i, 1
    vec1(claw_i,claw_j) = vec2(claw_i, claw_j) + 10    
  END DO
END DO
END SUBROUTINE vector_add
\end{lstlisting}


\subsection{remove 2}
\label{remove2}

\textbf{Original code}
\begin{lstlisting}
DO k=1, kend
  DO i=1, iend
    ! loop #1 body here
  END DO

  !$claw remove
  PRINT*, k
  PRINT*, k+1
  !$claw end remove

  DO i=1, iend
    ! loop #2 body here
  END DO
END DO
\end{lstlisting}


\textbf{Transformed code}
\begin{lstlisting}
DO k=1, kend
  DO i=1, iend
    ! loop #1 body here
  END DO

  DO i=1, iend
    ! loop #2 body here
  END DO
END DO
\end{lstlisting}
