\subsection{Single Column Abstraction (SCA)}

The \lstinline!sca! directive is used to parallelize a single column based
algorithm. In weather prediction models, the physical parametrizations are
column independent problems. This means that the algorithm to compute the
different output fields, can be defined with as a single do statement
iterating over the vertical dimension and there is no horizontal dependency.
The \lstinline!sca! directive is then used to parallelize efficiently
this column algorithm over the domain (horizontal dimensions).
The directive is local to a subroutine. Therefore, as a pre-condition, the
column algorithm must be enclosed in a module subroutine.

\lstset{emph={dim_id,lower_bound,upper_bound,var_1,var_2}, emphstyle=\itshape}
\begin{lstlisting}
!$claw define dimension dim_id(lower_bound:upper_bound) &
[!$claw define dimension dim_id(lower_bound:upper_bound) &] ...
!$claw sca [data(var_1[,var_2] ...)] [over (dim_id|:[,dim_id|:] ...)]
\end{lstlisting}

\textbf{Options and details}
\begin{itemize}
\item \textit{define dimension}: Define a new dimension on which the column
model will be parallelized. The lower bound and upper bound of the defined
dimension can be either an integer constant or an integer variable. If a
variable is given, it will be added in-order to the signature of the subroutine
as an \lstinline!INTENT(IN)! parameter.
\item \textit{data (optional)}: Define a list of variables to be promoted and
bypass the automatic promotion deduction.
\item \textit{over (optional)}: Define the location of the new dimensions in
promoted variables. The \lstinline!:! symbol reflects the position of the
current dimensions before the promotion.
\end{itemize}

If the \lstinline!data! clause is omitted, the automatic promotion of variables
is done as follows:
\begin{itemize}
\item Array variables declared with the \lstinline!INOUT! or \lstinline!OUT!
intents are automatically promoted with the defined dimensions.
In function of the target, intermediate scalar or array variables may also be
promoted.
\item Scalar/array variables placed on the left hand-side of an assignment
statement will be promoted if the right hand-side references any variables
already promoted. If the \lstinline!over! clause is omitted, the dimensions are
added on the left-side following the the definition order if there is multiple
new dimensions. The \lstinline!over! clause allows to change this behavior and
to place the new dimensions where needed.
\end{itemize}

As the subroutine/function signature is updated by the \lstinline!sca!
directive with new dimensions, the call graph that leads to this
subroutine/function must be updated as well. For this purpose, the clause
\lstinline!forward! allows to replicated the changes along the call graph.
Each subroutine/function along the call graph must be decorated with this
directive.

\begin{lstlisting}
! Call of a sca subroutine
!$claw sca forward
CALL one_column_fct(x,y,z)

! Call of a sca function
!$claw sca forward
result = one_column_fct(x,y,z)
\end{lstlisting}

The different variables declared in the current function/subroutine might be
promoted if the call requires it. In this case, the array references in
the function/subroutine are updated accordingly. This can lead to extra
promotion of local scalar variables or arrays when needed.

The root function can be encapsulated in a parallel region if the 
\lstinline|parallel| clause is added to the \lstinline|sca forward| directive.

The root function call in the call graph might be an iteration over several
column to reproduce the algorithm on the grid for testing purpose. If the
directive with the \lstinline!forward! clause is placed just before one or
several nested do statements with a function call, the corresponding do
statements will be removed and the function call updated accordingly.

\begin{lstlisting}
!$claw sca forward
DO i=istart, iend
  CALL one_columne_fct(x,y(i,:),z(i,:))
END DO
\end{lstlisting}

The root function call can also introduce blocking. For this, 
the \lstinline|blocked| clause must be added to the directive. 
The \lstinline|blocked| clause takes an argument that can be a positive integer
or an \lstinline|INTEGER| variable.


\begin{lstlisting}
!$claw sca forward blocked(bsize)
DO p = 1, nproma     
  CALL compute_column(nz, q(p,:), t(p,:), z(p,:))                                                                                             
END DO 
\end{lstlisting}

The transformed code will automatically compute block sizes and split the 
horizontal dimension accordingly.
\begin{lstlisting}
p_blocks = <number of blocks as a function of "bsize">                                                                                                                                                                                                                                                 
DO p_i = 1, p_blocks                                                                                                                                                                                                                                                          
   p0 = <first index of block in global array>                                                                                                                                                                                                                                
   p1 = <final index of block in global array>                                                                                                                                                                                                                                
   CALL compute_column ( nz , q (p0:p1 , : ) , t (p0:p1, : ) , z (p0:p1, : ), &
     nproma=p1-p0 )   
END DO
\end{lstlisting}

Computation of the number of blocks is given by the implementation. User can 
also provide a function to perform this computation. This function must takes an
\lstinline|INTEGER| argument and return an \lstinline|INTEGER| value.  

To provide this function, the user can use the 
\lstinline|block-fct(<function-name>)| clause.

\textbf{Code example}\\
\label{parallelize1}
Simple example of a column model wrapped into a subroutine and parallelized with
CLAW.

Original code
\lstinputlisting[frame=rlbt,language=Fortran]
  {./code_samples/parallelize1/main.f90}
\lstinputlisting[frame=rlbt,language=Fortran]
  {./code_samples/parallelize1/mo_column.f90}

Transformed code\\
Only the code of the module is shown as it is the only one which changes.\\

CLAW generated code for GPU target with OpenACC accelerator directive language.
\lstinputlisting[frame=rlbt,language=Fortran]
  {./code_samples/parallelize1/mo_column_gpu_openacc.f90}

CLAW generated code for CPU target with OpenMP accelerator directive language.
\lstinputlisting[frame=rlbt,language=Fortran]
  {./code_samples/parallelize1/mo_column_cpu_openmp.f90}

Compilation with the CLAW FORTRAN Compiler
\begin{lstlisting}[language=bash]
# For GPU target with OpenACC directive
clawfc --target=gpu --directive=openacc -o mo_column.gpu_acc.f90 mo_column.f90
clawfc --target=gpu --directove=openacc -o main.claw.f90 main.f90

# For CPU target without directive
clawfc -t=cpu -d=none -o mo_column.cpu_omp.f90 mo_column.f90
clawfc -t=cpu -d=none -o main.claw.f90 main.f90
\end{lstlisting}

\subsubsection{SCA directive in ELEMENTAL function}
Single Column Abstraction can be used within an ELEMENTAL function or
subroutine. Depending on the desired target, the function/subroutine can be
transformed to fit the programming paradigm. Once transformed, the
function/subroutine is not garuanteed to be PURE or ELEMENTAL anymore.

\subsubsection{Automatic Data Movement Managment}
Some targets need to take care of data movement between an host and a device.
This data movment can be handled by CLAW using the followings clauses on
\lstinline|!$claw sca forward| directive.%

\begin{itemize}
  \item \lstinline|create|: the create clause will generate the appropriate
        data allocation/deallocation before and after the
        \lstinline|sca forward| directive.
  \item \lstinline!update[(in|out)]!: the update clause will generate the 
        appropriate update before and/or after the \lstinline|sca forward| 
        directive. The update clause accepts \lstinline|in| or \lstinline|out| 
        optional options.
\end{itemize}

This clauses have no effect when the transformation target does not need a
data movement management.

\subsubsection{SCA model configuration}
Instead of defining dimension per subroutine or function, one can define them 
for the whole model. These definitions have to be done in a specific 
configuration file as detailled in this section. The promotion of variables are 
based on defined layout also specified in the same configuration file.

When using a model configuration file, the is simplified to the minimum required
as shown below:
\lstset{emph={layout_id}, emphstyle=\itshape}
\begin{lstlisting}
!$claw sca [layout(layout_id)]
\end{lstlisting}

In addition to the \lstinline|sca| directive, a new directive is used to specify
which variables are "model data".
Model data are variables that will be promoted during the transformation. The 
\lstinline|model-data| directive is a block directive and has to be placed in 
the declaration section of the subroutine/function. One or more 
\lstinline|model-data| block can be declared in a subroutine/function.

\begin{lstlisting}
!$claw model-data

! Variable declaration

!$claw end model-data
\end{lstlisting}

  
The specification of the model configuration file are listed below:
\begin{itemize}
  \item The model configuration file follows the TOML file format. 
  \item The configuration is spearated in three tables: model, dimensions, 
        layouts.
\end{itemize}

\textbf{Model table}\\
The model table contains all global information about the model and its 
configuration.

\textbf{Dimensions table}\\
The dimensions table contains all dimension definitions. Each definition can 
have the information listed here: 
\begin{itemize}
  \item \textit{id}: the identifier of the dimension definition. It is unique 
        in the model configuration.
  \item \textit{size}: the size of the dimension as a lower and upper bound. 
        The lower bound can be omitted and the default value 1 will be assigned. 
        Upper bound is mandatory.
  \item \textit{iteration}: iteration information of the dimension defined with 
        a lower and upper bound as well as a step. All information are optional. 
        If omitted, lower and upper bound are copied from size information and 
        the default value 1 is assigned to the step.      
\end{itemize}


\textbf{Layouts table}\\
The layout table contains all the different layout that can be applied for a 
specific target or a specific subroutine/function.
\begin{itemize}
  \item \textit{name}: the identifier of the layout definition. It is unique in 
        the model configuration. \lstinline|default| is the only mandatory 
        layout definiton.
  \item \textit{position}: list of dimension definition that defined the layout. 
        The semi-colon (:) define already existing definition.
\end{itemize}

\lstinputlisting[frame=rlbt,language=python,
      caption={Example of SCA model configuration file.}]
      {./code_samples/sca/model.toml}