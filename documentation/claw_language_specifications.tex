\documentclass{article}
\usepackage{geometry}
\usepackage{xcolor}
\usepackage{listings}
\usepackage{courier}
\lstset{%
  backgroundcolor=\color{gray!20},
  breaklines=true,
  basicstyle=\small\ttfamily,
  keywordstyle=\small\ttfamily\bfseries,
  commentstyle=\ttfamily,
  frame=single, 
  language=Fortran,
  numbers=left,
  showspaces=false,
  showstringspaces=false
}
\lstdefinestyle{noumbers}
{numbers=none}
\usepackage{enumerate}

\title{CLAW language specification}
\author{Center for Climate System Modeling (C2SM)\\http://www.c2sm.ethz.ch\\Valentin Clement}
\date{April 21, 2016\\\vspace{1em}v0.2}

\setlength\parindent{0pt}
\setlength{\parskip}{1em}

\begin{document}
\maketitle


\tableofcontents

\section{General information about the CLAW language}
The directives are either local or global.

\begin{itemize}
\item Local directive: those directives have a limited impact on a local block of
code (for example, only in a subroutine)
\item Global directive: those directives can have an impact on the whole
application.
\end{itemize}

This language is separated in the followings sections:
\begin{itemize}
\item CLAW abstraction
\item Loop transformation
\item Accelerator abstractions/helpers
\item Utilities
\end{itemize}

\subsection{Line continuation}
CLAW directives can be defined on several line. The syntax is described in the
listing below:

\begin{lstlisting}
!$claw directive clause &
!$claw clause
\end{lstlisting}


\subsection{Interpretation order of the CLAW directives}
The claw directives can be combined together. For example, loop-fusion and
loop-interchange can be used together in a group of nested loops.

The standard interpretation order of the directives is the following:

\begin{enumerate}
\item remove
\item array-transform
\item loop-extract
\item loop-fusion
\item loop-hoist
\item loop-interchange
\item call
\item kcache
\item formatting transformation (internal transformation only)
\end{enumerate}

Users must be aware that directives transformation are applied sequentially and
therefore, a transformation can be performed on code that is already transformed.

\section{CLAW abstraction}

\subsection{K caching (column caching)}
\begin{lstlisting}
!$claw kcache data((var[,var] ...)) [(offset[ offset] ...)]] [init] [private]
\end{lstlisting}

In memory-bound problem, it might be useful to cache array values used several times during the current loop iteration or for the next one.

The \textbf{kcache} directive is applied in a limited do statement block. It will cache the corresponding assigned value and update
the array index in the block according to the given plus/minus offset.

If the offset value is omitted, there are inferred from the dimension of the variable to be cached and set to 0.

\textbf{Options and details}
\begin{itemize}
\item \textit{data}: The data clause specifies which array will be impacted by the column caching transformation. 
\item \textit{offset}: Integer value separated by a space that represents the offset at each dimension. 
\item \textit{init}: If the \textit{init} clause is specified, the cache variable will be initialized with the corresponding array value at the given offset during the 1st loop iteration. 
\item \textit{private}: it declares that a copy of each item on the list will be created
for each parallel gang on the accelerator. The list is the one specified on
the `data` clause or inferred with assignment cache.
\end{itemize}




\subsection{On the fly computation (array access to function call)}
\begin{lstlisting}
!$claw call array_name=function_call(arg_list)
\end{lstlisting}

Sometimes, replacing access to pre-computed arrays with computation on the fly
can increase the performance. It can reduce the memory access for memory-bound
kernel and exploit some unused resource to execute the computation.

\section{Loop transformation}
\subsection{Loop interchange/reordering}
\begin{lstlisting}
!$claw loop-interchange [(induction_var[, induction_var] ...)]
\end{lstlisting}

Loop reordering is a common transformation applied on do statements when adding parallelization. This transformation is mainly used to improve the data locality and avoid the need to transpose the arrays.

The \textbf{loop-interchange} directive must be placed just before the first do statement composing the nested do statements group. With a group of two nested do statements, the induction variable and iteration range of the first statement is swap with the information of the second statement (see example \ref{loop-interchange1}). 

If the list of induction variables is specified, the do statements are reordered according to the order defined in the list (see example \ref{loop-interchange2}).

\textbf{Options and details}\\
\textit{induction\_var}: induction variable of the do statement. The list gives the new order of the do statement after the transformation. For group of 2 nested do statements, this information is optional. 

\textbf{Behavior with other directives}\\
When the loops to be reordered are decorated with additional directives, those directives stay in place during the code transformation. In other words, they are not reordered together with the loops (see example \ref{loop-interchange3}).

\textbf{Limitations}\\
Currently, the loop-interchange directive is limited to 3 level of nested do statements. More nested levels can be declared but the transformation is limited to the first 3 levels from the directive declaration.

\subsection{Loop jamming/fusion}
\begin{lstlisting}
!$claw loop-fusion [group(group_id)] [collapse(n)]
\end{lstlisting}

Loop jamming or fusion is used to merge 2 or more do statements together. Sometimes, the
work performed in a do statement is too small to create significant impact on
performance when it is parallelized. Merging some do statements together create bigger
blocks (kernels) to be parallelized.

If the \textit{group} clause is not specified, all the do statements decorated with the directive in the
same block will be merged together as a single do statement.

If the \textit{group} clause is specified, the loops are merged in-order with do statements sharing 
the same group in the current structured block.

The \textit{collapse} clause is used to specify how many tightly nested do statements are
associated with the \textbf{loop-fusion} construct. The argument to the \textit{collapse}
clause is a constant positive integer. If no \textit{collapse} clause
is specified, only the immediate following do statement is associated with the
\textbf{loop-fusion} construct.

\textbf{Options and details}
\begin{itemize}
\item \textit{group\_id}: A string label that identify a group of do statement in a structured block. 
\item \textit{n}: A constant positive integer.
\end{itemize}

\textbf{Behavior with other directives}\\
When the do statement to be merged are decorated with other directives, only the
directives on the first do statement of the merge group are kept in the transformed
code.

\textbf{Limitations}\\
All the do statement within a group must share the same iteration range. If the
\textit{collapse} clause is specified, the loops must share the same iteration range at the
corresponding depth (see example \ref{loop-fusion4}).

\subsection{Loop extraction}
\begin{lstlisting}
!$claw loop-extract range(range) [map(var[,var]...:mapping[,mapping]...) [map(var[,var]...:mapping[,mapping]...)] ...] [fusion [group(group_id)]] [parallel] [acc(directives)]
\end{lstlisting}

Loop extraction can be performed on a subroutine/function call. The do statement corresponding
to the defined iteration range is extracted from the subroutine/function and is wrapped
around the subroutine/function call. In the transformation, a copy of the subroutine/function
is created with the corresponding transformation (demotion) for the parameters.


\textbf{Options and details}
\begin{enumerate}
\item \textit{range}: Correspond to the iteration range of the loop to be extracted.
  Notation \lstinline!i = istart, iend, istep!
\item \textit{map}: Define the mapping of variable that are demoted during the loop
  extraction. As seen in the example 1, the two parameters (1 dimensional array)
  are mapped to a scalar with the induction variable \textit{i}.
  \begin{enumerate}
    \item The \textit{var} can be defined as two parts variable (e.g. \lstinline!a/a1!). The
    first part is the function call part and refers to the variable as it is
    defined in the function call. The second part is the function definition
    part and refers to the name of the variable to be mapped as it defined in
    the function declaration. If a \textit{var} is defined as a single part variable,
    the same name is used for both the function call and function definition
    part.
    \item The \textit{mapping} can be defined as two parts variable (e.g. \lstinline!i/i1!). The
    first part is the function call part and refers to the mapping variable as
    it is defined in the function call. The second part is the function
    definition part and refers to the name of the mapping variable as it defined
    in the function declaration. If a \textit{mapping} is defined as a single part
    mapping variable, the same name is used for both the function call and
    function definition part.
    \end{enumerate}
\item \textit{fusion}: Allow the extracted loop to be merged with other loops.
Options are identical with the \lstinline!loop-fusion! directive.
\item \textit{parallel}: Wrap the extracted loop in a parallel region.
\item \textit{acc}: Add the accelerator directives to the extracted loop.
\end{enumerate}

If the directive \textbf{loop-extract} is used for more than one call to the same
subroutine, the extraction can generate 1 to N dedicated subroutines.

\textbf{Behavior with other directives}\\
Accelerator directives are generated by the transformation. If the function call was decorated 
with accelerator directives prior to the transformation, those directives stay in place. 

\subsection{Loop hoisting}
\begin{lstlisting}
!$claw loop-hoist(induction_var[[, induction_var] ...]) [reshape(array_name(target_dimension[,kept_dimension]))] [interchange [(induction_var[[, induction_var] ...])]]

  ! structured block of code
  
!$claw end loop-hoist
\end{lstlisting}

The \textbf{loop-hoist} directive allows nested loops in a defined structured block to
be merged together and to hoist the beginning of those nested loop just after
the directive declaration. Loops with different lower-bound indexes
can also be merged with the addition of an \lstinline!IF! statement. This feature works 
only when the lower-bound are integer constant. 


\textbf{Options and details}
\begin{enumerate}
\item \textit{interchange}: Allow the group of hoisted loops to be reordered.
Options are identical with the \textbf{loop-interchange} directive.
\item \textit{reshape}: Reshape arrays to scalar or to array with fewer dimensions. 
\end{enumerate}


\section{Accelerator abstractions/helpers}

\subsection{Array notation to do statements}

\begin{lstlisting}
!$claw array-transform [induction(name [[,] name]...)] [fusion [group(group_id)]] [parallel] [acc([clause [[,] clause]...])]

  ! array notation assignment(s) 
  
[!$claw end remove]
\end{lstlisting}

Computations using the array notation are not suitable to be parallelized with
language like OpenACC. The \textbf{array-transform} directive allows to transform those
notation with the corresponding loops which are more suitable for
parallelization.

The goal of this directive is to pass from an array notation assignment like
this:

\begin{lstlisting}
A(1:n) = A(1+m:n+m) + B(1:n) * C(n+1:n+n)
\end{lstlisting}

To a do loop statement like this:

\begin{lstlisting}
DO i=1,n
  A(i) = A(i+m) + B(i) * C(n+i)
END DO
\end{lstlisting}

If the directive is used as a block directive, the assignments are wrapped in
a single do statement if their induction range match.

\begin{lstlisting}
A(1:n) = A(1+m:n+m) + B(1:n) * C(n+1:n+n)
B(1:n) = B(1:n) * 0.5
\end{lstlisting}

To

\begin{lstlisting}
DO i=1,n
  A(i) = A(i+m) + B(i) * C(n+i)
  B(i) = B(i) * 0.5
END DO
\end{lstlisting}


\textbf{Options and details}
\begin{enumerate}
\item \textit{induction}: Allow to name the induction variable.
\item \textit{fusion}: Allow the extracted loop to be merged with other loops. Options 
are identical with the `loop-fusion` directive
\item \textit{parallel}: Wrap the extracted loop in a parallel region.
\item \textit{acc}: Define accelerator clauses that will be applied to the generated loops.
\end{enumerate}

\textbf{Behavior with other directives}\\
Directives declared before the \textbf{array-transform} directive will be kept in the
generated code.

\section{Utilities}

\subsection{Remove}
\begin{lstlisting}
!$claw remove

  ! code block
  
[!$claw end remove]
\end{lstlisting}

The \textbf{remove} directive allows the developer to remove section of code
during the transformation process. 

\textbf{Options and details}\\
If the directive is directly followed by a structured block (\lstinline!IF! or \lstinline!DO!), the
end directive is not mandatory (see example \ref{remove1}). In any other cases, the end
directive is mandatory.

\textbf{Behavior with other directives}\\
None

\textbf{Limitations}\\
None


\appendix

\section{Code examples}

\subsection{kcache 1}
\label{kcache1}

\textbf{Original code}
\begin{lstlisting}
!$claw kcache 0 -1
ztu6(j1,ki3sc) = x * y * z
DO j3 = ki3sc+1, ki3ec
  var1 = x * y - ztu6(j1, j3-1)

  ztu6(j1,j3) = x * y * z + var1
END DO
\end{lstlisting}


\textbf{Transformed code}
\begin{lstlisting}
! CLAW kcache
ztu6_km1 = x * y * z
ztu6(j1,ki3sc) = ztu6_km1
DO j3 = ki3sc+1, ki3ec
  var1 = x * y - ztu6_km1

  ztu6_km1 = x * y * z + var1
  ztu6(j1,j3) = x * y * z + var1
END DO
\end{lstlisting}




\subsection{loop-interchange 1}
\label{loop-interchange1}
Example with 2 levels of loop. 

\textbf{Original code}
\begin{lstlisting}
!$claw loop-interchange
DO i=1, iend
  DO k=1, kend
    ! loop body here
  END DO
END DO
\end{lstlisting}

\textbf{Transformed code}
\begin{lstlisting}
! CLAW transformation (loop-interchange i < -- > k)
DO k=1, kend
  DO i=1, iend
    ! loop body here
  END DO
END DO
\end{lstlisting}

\subsection{loop-interchange 2}
\label{loop-interchange2}
Example with 3 levels of loop. 

\textbf{Original code}
\begin{lstlisting}
!$claw loop-interchange (k,i,j)
DO i=1, iend     ! loop at depth 0
  DO j=1, jend   ! loop at depth 1
    DO k=1, kend ! loop at depth 2
      ! loop body here
    END DO
  END DO
END DO
\end{lstlisting}

\textbf{Transformed code}
\begin{lstlisting}
! CLAW transformation (loop-interchange (k,i,j))
DO k=1, kend       ! loop at depth 2
  DO i=1, iend     ! loop at depth 0
    DO j=1, jend   ! loop at depth 1
      ! loop body here
    END DO
  END DO
END DO
\end{lstlisting}

\subsection{loop-interchange 3}
\label{loop-interchange3}
Example with OpenACC directives. 

\textbf{Original code}
\begin{lstlisting}
!$acc parallel
!$acc loop gang
!$claw loop-interchange
DO i=1, iend
  !$acc loop vector
  DO k=1, kend
    ! loop body here
  END DO
END DO
!$acc end parallel
\end{lstlisting}

\textbf{Transformed code}
\begin{lstlisting}
! CLAW transformation (loop-interchange i < -- > k)
!$acc parallel
!$acc loop gang
DO k=1, kend
  !$acc loop vector
  DO i=1, iend
    ! loop body here
  END DO
END DO
!$acc end parallel
\end{lstlisting}

\subsection{loop-fusion 1}
Example without clauses. 

\textbf{Original code}
\begin{lstlisting}
DO k=1, iend
  !$claw loop-fusion
  DO i=1, iend
    ! loop #1 body here
  END DO

  !$claw loop-fusion
  DO i=1, iend
    ! loop #2 body here
  END DO
END DO
\end{lstlisting}

\textbf{Transformed code}
\begin{lstlisting}
DO k=1, iend
  ! CLAW transformation (loop-fusion same block group)
  DO i=1, iend
    ! loop #1 body here
    ! loop #2 body here
  END DO
END DO
\end{lstlisting}

\subsection{loop-fusion 2}
\label{loop-fusion2}
Example using the \textit{group} clause.

\textbf{Original code}
\begin{lstlisting}
DO k=1, iend
  !$claw loop-fusion group(g1)
  DO i=1, iend
    ! loop #1 body here
  END DO

  !$claw loop-fusion group(g1)
  DO i=1, iend
    ! loop #2 body here
  END DO

  !$claw loop-fusion group(g2)
  DO i=1, jend
    ! loop #3 body here
  END DO

  !$claw loop-fusion group(g2)
  DO i=1, jend
    ! loop #4 body here
  END DO
END DO
\end{lstlisting}


\textbf{Transformed code}
\begin{lstlisting}
DO k=1, iend
  ! CLAW transformation (loop-fusion group g1)
  DO i=1, iend
    ! loop #1 body here
    ! loop #2 body here
  END DO

  ! CLAW tranformation (loop-fusion group g2)
  DO i=1, jend
    ! loop #3 body here
    ! loop #4 body here
  END DO
END DO
\end{lstlisting}

\subsection{loop-fusion 3}
\label{loop-fusion3}
Example showing behavior with other directives (OpenACC). 

\textbf{Original code}
\begin{lstlisting}
!$acc parallel
!$acc loop gang
DO k=1, iend
  !$acc loop seq
  !$claw loop-fusion
  DO i=1, iend
    ! loop #1 body here
  END DO

  !$acc loop vector
  !$claw loop-fusion
  DO i=1, iend
    ! loop #2 body here
  END DO
END DO
!$acc end parallel
\end{lstlisting}

\textbf{Transformed code}
\begin{lstlisting}
!$acc parallel
!$acc loop gang
DO k=1, iend
  ! CLAW transformation (loop-fusion same block group)
  !$acc loop seq
  DO i=1, iend
    ! loop #1 body here
    ! loop #2 body here
  END DO
END DO
!$acc end parallel
\end{lstlisting}

\subsection{loop-fusion 4}
\label{loop-fusion4}
Example using the \textit{collapse} clause. 

\textbf{Original code}
\begin{lstlisting}
DO k=1, iend
  !$claw loop-fusion collapse(2)
  DO i=0, iend
    DO j=0, jend
      ! nested loop #1 body here
    END FO
  END DO

  !$claw loop-fusion collapse(2)
  DO i=0, iend
    DO j=0, jend
      ! loop #2 body here
    END DO
  END DO
END DO
\end{lstlisting}


\textbf{Transformed code}
\begin{lstlisting}
DO k=1, iend
  ! CLAW transformation (loop-fusion collapse(2))
  DO i=0, iend
    DO j=0, jend
      ! nested loop #1 body here
      ! nested loop #2 body here
    END DO
  END DO
END DO
\end{lstlisting}



\subsection{loop-extract 1}
\label{loop-extract1}
Example without additional clauses. 

\textbf{Original code}
\begin{lstlisting}
PROGRAM main
  !$claw loop-extract(i=istart,iend) map(value1,value2:i)
  CALL xyz(value1, value2)
END PROGRAM main

SUBROUTINE xyz(value1, value2)
  REAL, INTENT (IN) :: value2(x:y), value2(x:y)

  DO i = istart, iend
    ! some computation with value1(i) and value2(i) here
  END DO
END SUBROUTINE xyz
\end{lstlisting}


\textbf{Transformed code}
\begin{lstlisting}
PROGRAM main
  !CLAW extracted loop
  DO i = istart, iend
    CALL xyz_claw(value1(i), value2(i))
  END DO
END PROGRAM main

SUBROUTINE xyz(value1, value2)
  REAL, INTENT (IN) :: value2(x:y), value2(x:y)

  DO i = istart, iend
    ! some computation with value1(i) and value2(i) here
  END DO
END SUBROUTINE xyz

!CLAW extracted loop new subroutine
SUBROUTINE xyz_claw(value1, value2)
  REAL, INTENT (IN) :: value1, value2
  ! some computation with value1 and value2 here
END SUBROUTINE xyz_claw
\end{lstlisting}

\subsection{loop-extract 2}
\label{loop-extract2}
Example with \textit{fusion} clause. 

\textbf{Original code}
\begin{lstlisting}
PROGRAM main
  !$claw loop-extract(i=istart,iend) map(value1,value2:i) fusion group(g1)
  CALL xyz(value1, value2)

  !$claw loop-fusion group(g1)
  DO i = istart, iend
    ! some computation here
    print*,'Inside loop', i
  END DO
END PROGRAM main

SUBROUTINE xyz(value1, value2)
  REAL, INTENT (IN) :: value2(x:y), value2(x:y)

  DO i = istart, iend
    ! some computation with value1(i) and value2(i) here
  END DO
END SUBROUTINE xyz
\end{lstlisting}

\textbf{Transformed code}
\begin{lstlisting}
PROGRAM main
  !CLAW extracted loop
  DO i = istart, iend
    CALL xyz_claw(value1(i), value2(i))
    ! some computation here
    print*,'Inside loop', i
  END DO
END PROGRAM main

SUBROUTINE xyz(value1, value2)
  REAL, INTENT (IN) :: value2(x:y), value2(x:y)

  DO i = istart, iend
    ! some computation with value1(i) and value2(i) here
  END DO
END SUBROUTINE xyz

!CLAW extracted loop new subroutine
SUBROUTINE xyz_claw(value1, value2)
  REAL, INTENT (IN) :: value1, value2
  ! some computation with value1 and value2 here
END SUBROUTINE xyz_claw
\end{lstlisting}


\subsection{loop-extract 3}
\label{loop-extract3}
Example with more complicated mapping. 

\textbf{Original code}
\begin{lstlisting}
PROGRAM main
  !$claw loop-extract(i=istart,iend) map(value1,value2:i/j)
  CALL xyz(value1, value2)
END PROGRAM main

SUBROUTINE xyz(value1, value2, j)
  INTGER, INTENT(IN) :: j
  REAL  , INTENT(IN) :: value2(x:y), value2(x:y)

  DO i = istart, iend
    ! some computation with value1(j) and value2(j) here
  END DO
END SUBROUTINE xyz
\end{lstlisting}

\textbf{Transformed code}
\begin{lstlisting}
PROGRAM main
  !CLAW extracted loop
  DO i = istart, iend
    CALL xyz_claw(value1(i), value2(i))
  END DO
END PROGRAM main

SUBROUTINE xyz(value1, value2, j)
  INTGER, INTENT(IN) :: j
  REAL  , INTENT(IN) :: value2(x:y), value2(x:y)

  DO i = istart, iend
    ! some computation with value1(j) and value2(j) here
  END DO
END SUBROUTINE xyz

!CLAW extracted loop new subroutine
SUBROUTINE xyz_claw(value1, value2, j)
  INTGER, INTENT(IN) :: j
  REAL, INTENT (IN) :: value1, value2
  ! some computation with value1 and value2 here
END SUBROUTINE xyz_claw
\end{lstlisting}

\subsection{loop-hoist 1}
\label{loop-hoist1}
Example without clauses. 

\textbf{Original code}
\begin{lstlisting}
!$acc parallel loop gang vector collapse(2)
DO jt=1,jtend
  !$claw loop-hoist(j,i) interchange
  IF ( .TRUE. ) CYCLE
    ! outside loop statement
  END IF
  DO j=1,jend
    DO i=1,iend
      ! first nested loop body
    END DO
  END DO
  DO j=2,jend
    DO i=1,iend
      ! second nested loop body
    END DO
  END DO
  !$claw end loop-hoist
END DO
!$acc end parallel
\end{lstlisting}


\textbf{Transformed code}
\begin{lstlisting}
!$acc parallel loop gang vector collapse(2)
DO jt=1,jtend
  DO i=1,iend
    DO j=1,jend
      IF ( .TRUE. ) CYCLE
        ! outside loop statement
      END IF
      ! first nested loop body
      IF(j>1) THEN
        ! second nested loop body
      END IF
    END DO
  END DO
END DO
!$acc end parallel
\end{lstlisting}


\subsection{array-transform 1}
\label{array-transform1}
Example with array notation following the pragma statement. 

\textbf{Original code}
\begin{lstlisting}
SUBROUTINE vector_add
  INTEGER :: i = 10
  INTEGER, DIMENSION(0:9) :: vec1

  !$claw array-transform
  vec1(0:i) = vec1(0:i) + 10;
END SUBROUTINE vector_add
\end{lstlisting}

\textbf{Transformed code}
\begin{lstlisting}
SUBROUTINE vector_add
  INTEGER :: claw_i
  INTEGER :: i = 10
  INTEGER, DIMENSION(0:9) :: vec1

  ! CLAW transformation array notation to do loop
  DO claw_i = 0, i
    vec1(claw_i) = vec1(claw_i) + 10;
  END DO
END SUBROUTINE vector_add
\end{lstlisting}


\subsection{array-transform 2}
\label{array-transform2}
Example with \textit{induction} and \textit{acc} clauses

\textbf{Original code}
\begin{lstlisting}
SUBROUTINE vector_add
  INTEGER :: i = 10
  INTEGER, DIMENSION(0:9) :: vec1

  !$acc parallel
  !$claw array-transform induction(myinduc) acc(loop)
  vec1(0:i) = vec1(0:i) + 10;

  !$claw array-transform acc(loop)
  vec1(0:i) = vec1(0:i) + 1;
  !$acc end parallel
END SUBROUTINE vector_add
\end{lstlisting}


\textbf{Transformed code}
\begin{lstlisting}
SUBROUTINE vector_add
  INTEGER :: claw_i
  INTEGER :: myinduc
  INTEGER :: i = 10
  INTEGER, DIMENSION(0:9) :: vec1

  !$acc parallel

  ! CLAW transformation array notation vec1(0:i) to do loop
  !$acc loop
  DO myinduc = 0, i
    vec1(myinduc) = vec1(myinduc) + 10;
  END DO

  ! CLAW transformation array notation vec1(0:i) to do loop
  !$acc loop
  DO claw_i = 0, i
    vec1(claw_i) = vec1(claw_i) + 1;
  END DO
  
  !$acc end parallel
END SUBROUTINE vector_add
\end{lstlisting}

\subsection{array-transform 3}
\label{array-transform3}
Example with fusion clause
 
\textbf{Original code}
\begin{lstlisting}
SUBROUTINE vector_add
  INTEGER :: i = 10
  INTEGER, DIMENSION(0:9) :: vec1
  INTEGER, DIMENSION(0:9) :: vec2

  !$claw array-transform induction(claw_i) fusion
  vec1(0:i) = vec1(0:i) + 10;

  !$claw array-transform induction(claw_i) fusion
  vec2(0:i) = vec2(0:i) + 1;
END SUBROUTINE vector_add
\end{lstlisting}


\textbf{Transformed code}
\begin{lstlisting}
SUBROUTINE vector_add
  INTEGER :: claw_i
  INTEGER :: i = 10
  INTEGER, DIMENSION(0:9) :: vec1
  INTEGER, DIMENSION(0:9) :: vec2

  ! CLAW transformation array notation vec1(0:i) to do loop
  ! CLAW transformation array notation vec2(0:i) to do loop
  ! CLAW transformation fusion
  DO claw_i=0, i
    vec1(claw_i) = vec1(claw_i) + 10;
    vec2(claw_i) = vec2(claw_i) + 1;
  END DO
END SUBROUTINE vector_add
\end{lstlisting}


\subsection{array-transform 4}
\label{array-transform4}
Example with 2-dimensional arrays

\textbf{Original code}
\begin{lstlisting}
SUBROUTINE vector_add
INTEGER :: i = 10
INTEGER, DIMENSION(0:10,0:10) :: vec1
INTEGER, DIMENSION(0:10,0:10) :: vec2

vec1(0:i,0:i) = 0;
vec2(0:i,0:i) = 100;

!$claw array-transform
vec1(0:i,0:i) = vec2(0:i,0:i) + 10
END SUBROUTINE vector_add
\end{lstlisting}

\textbf{Transformed code}
\begin{lstlisting}
SUBROUTINE vector_add
INTEGER :: i = 10
INTEGER, DIMENSION(0:10,0:10) :: vec1
INTEGER, DIMENSION(0:10,0:10) :: vec2

vec1(0:i,0:i) = 0;
vec2(0:i,0:i) = 100;

DO claw_i = 0, i, 1
  DO claw_j = 0, i, 1
    vec1(claw_i,claw_j) = vec2(claw_i, claw_j) + 10    
  END DO
END DO
END SUBROUTINE vector_add
\end{lstlisting}

\subsection{remove 1}
\label{remove1}

\textbf{Original code}
\begin{lstlisting}
DO k=1, kend
  DO i=1, iend
    ! loop #1 body here
  END DO

  !$claw remove
  IF (k > 1) THEN
    PRINT*, k
  END IF

  DO i=1, iend
    ! loop #2 body here
  END DO
END DO
\end{lstlisting}

\textbf{Transformed code}
\begin{lstlisting}
DO k=1, kend
  DO i=1, iend
    ! loop #1 body here
  END DO

  DO i=1, iend
    ! loop #2 body here
  END DO
END DO
\end{lstlisting}




\subsection{remove 2}
\label{remove2}

\textbf{Original code}
\begin{lstlisting}
DO k=1, kend
  DO i=1, iend
    ! loop #1 body here
  END DO

  !$claw remove
  PRINT*, k
  PRINT*, k+1
  !$claw end remove

  DO i=1, iend
    ! loop #2 body here
  END DO
END DO
\end{lstlisting}


\textbf{Transformed code}
\begin{lstlisting}
DO k=1, kend
  DO i=1, iend
    ! loop #1 body here
  END DO

  DO i=1, iend
    ! loop #2 body here
  END DO
END DO
\end{lstlisting}


\end{document}
