\subsection{Column model abstraction}

The \lstinline!parallelize! directive is used to parallelize a column based
algorithm. In weather prediction models, the physical parametrizations are
column independent problems. This means that the algorithm to compute the
different output fields, can be defined with as a single do statement
iterating over the vertical dimension and there is no horizontal dependency.
The \lstinline!parallelize! directive is then used to parallelize efficiently
this column algorithm over the domain (horizontal dimensions).
The directive is local to a subroutine. Therefore, as a pre-condition, the
column algorithm must be enclosed in a module subroutine.

\lstset{emph={dim_id,lower_bound,upper_bound,var_1,var_2}, emphstyle=\itshape}
\begin{lstlisting}
!$claw define dimension dim_id(lower_bound:upper_bound) &
[!$claw define dimension dim_id(lower_bound:upper_bound) &] ...
!$claw parallelize [data(var_1[,var_2] ...)] [over (dim_id|:[,dim_id|:] ...)]
\end{lstlisting}

\textbf{Options and details}
\begin{itemize}
\item \textit{define dimension}: Define a new dimension on which the column
model will be parallelized. The lower bound and upper bound of the defined
dimension can be either an integer constant or an integer variable. If a
variable is given, it will be added in-order to the signature of the subroutine
as an \lstinline!INTENT(IN)! parameter.
\item \textit{data (optional)}: Define a list of variables to be promoted and
bypass the automatic promotion deduction.
\item \textit{over (optional)}: Define the location of the new dimensions in
promoted variables. The \lstinline!:! symbol reflects the position of the
current dimensions before the promotion.
\end{itemize}

If the \lstinline!data! clause is omitted, the automatic promotion of variables
is done as follows:
\begin{itemize}
\item Array variables declared with the \lstinline!INOUT! or \lstinline!OUT!
intents are automatically promoted with the defined dimensions.
In function of the target, intermediate scalar or array variables may also be
promoted.
\item Scalar/array variables placed on the left hand-side of an assignment
statement will be promoted if the right hand-side references any variables
already promoted. If the \lstinline!over! clause is omitted, the dimensions are
added on the left-side following the the definition order if there is multiple
new dimensions. The \lstinline!over! clause allows to change this behavior and
to place the new dimensions where needed.
\end{itemize}

As the subroutine/function signature is updated by the \lstinline!parallelize!
directive with new dimensions, the call graph that leads to this
subroutine/function must be updated as well. For this purpose, the clause
\lstinline!forward! allows to replicated the changes along the call graph.
Each subroutine/function along the call graph must be decorated with this
directive.

\begin{lstlisting}
! Call of a parallelized subroutine
!$claw parallelize forward
CALL one_column_fct(x,y,z)

! Call of a parallelized function
!$claw parallelize forward
result = one_column_fct(x,y,z)
\end{lstlisting}

The different variables declared in the current function/subroutine might be
promoted if the call requires it. In this case, the array references in
the function/subroutine are updated accordingly. This can lead to extra
promotion of local scalar variables or arrays when needed.

The root function call in the call graph might be an iteration over several
column to reproduce the algorithm on the grid for testing purpose. If the
directive with the \lstinline!forward! clause is placed just before one or
several nested do statements with a function call, the corresponding do
statements will be removed and the function call updated accordingly.

\begin{lstlisting}
!$claw parallelize forward
DO i=istart, iend
  CALL one_columne_fct(x,y(i,:),z(i,:))
END DO
\end{lstlisting}


\textbf{Code example}\\
\label{parallelize1}
Simple example of a column model wrapped into a subroutine and parallelized with
CLAW.

Original code
\lstinputlisting[frame=rlbt,language=Fortran]{./code_samples/parallelize1/main.f90}
\lstinputlisting[frame=rlbt,language=Fortran]{./code_samples/parallelize1/mo_column.f90}

Transformed code\\
Only the code of the module is shown as it is the only one which changes.\\

CLAW generated code for GPU target with OpenACC accelerator directive language.
\lstinputlisting[frame=rlbt,language=Fortran]{./code_samples/parallelize1/mo_column_gpu_openacc.f90}

CLAW generated code for CPU target with OpenMP accelerator directive language.
\lstinputlisting[frame=rlbt,language=Fortran]{./code_samples/parallelize1/mo_column_cpu_openmp.f90}

Compilation with the CLAW FORTRAN Compiler
\begin{lstlisting}[language=bash]
# For GPU target with OpenACC directive
clawfc --target=gpu --directive=openacc -o mo_column.gpu_acc.f90 mo_column.f90
clawfc -o main.claw.f90 main.f90

# For CPU target with OpenMP directive
clawfc --target=gpu --directive=openmp -o mo_column.cpu_omp.f90 mo_column.f90
clawfc -o main.claw.f90 main.f90
\end{lstlisting}
