\subsection{Loop transformation}

% ------------------------------------------------------------------------------
%
% LOOP INTERCHANGE
%
% ------------------------------------------------------------------------------
\subsubsection{Loop interchange/reordering}
\lstset{emph={induction_var}, emphstyle=\itshape}
\begin{lstlisting}
!$claw loop-interchange [(induction_var[, induction_var] ...)]
\end{lstlisting}

Loop reordering is a common transformation applied on do statements when adding
parallelization. This transformation is mainly used to improve the data locality
and avoid the need to transpose arrays.

The \textbf{loop-interchange} directive must be placed just before the first do
statement composing the nested do statements group. With a group of two nested
do statements, the induction variable and iteration range of the first statement
is swapped with the information of the second do statement (see example
\ref{loop-interchange1}).

If the list of induction variables is specified, the do statements are reordered
according to the order defined in the list (see example
\ref{loop-interchange2}).

\textbf{Options and details}
\begin{enumerate}
\item \textit{induction\_var}: induction variable of the do statement. The list
gives the new order of the do statement after the transformation. For group of
2 nested do statements, this information is not necessary, as the new order is
deducted.
\end{enumerate}

\textbf{Behavior with other directives}\\
When the do statements to be reordered are decorated with additional directives,
those directives stay in place during the code transformation. In other words,
they are not reordered together with the do statements iteration ranges
(see example \ref{loop-interchange3}).

\textbf{Limitations}\\
Currently, the \textbf{loop-interchange} directive is limited to 3 level of
nested do statements. More nested levels can be declared but the transformation
is limited to the first 3 levels from the directive declaration.

\textbf{Code example}\\\label{loop-interchange1}
Example with 2 levels of loop.

Original code
\begin{lstlisting}[language=Fortran]
!$claw loop-interchange
DO i=1, iend
  DO k=1, kend
    ! loop body here
  END DO
END DO
\end{lstlisting}

Transformed code
\begin{lstlisting}[language=Fortran]
! CLAW transformation (loop-interchange i < -- > k)
DO k=1, kend
  DO i=1, iend
    ! loop body here
  END DO
END DO
\end{lstlisting}

More code examples in the appendix. Example with 3 nested do statements (see
example \ref{loop-interchange2}). Example with others directives (see example
\ref{loop-interchange3}).

% ------------------------------------------------------------------------------
%
% LOOP FUSION
%
% ------------------------------------------------------------------------------
\lstset{emph={group_id,n}, emphstyle=\itshape}
\subsubsection{Loop jamming/fusion}
\begin{lstlisting}
!$claw loop-fusion [group(group_id)] [collapse(n)]
\end{lstlisting}

Loop jamming or fusion is used to merge 2 or more do statements together.
Sometimes, the work performed in a single do statement is too small to create
significant impact on performance when it is parallelized. Merging some do
statements together create bigger blocks (kernels) to be parallelized.

If the \textit{group} clause is not specified, all the do statements decorated
with the directive in the same structured block and at the same depth in the AST
will be merged together as a single do statement (see code example in this
section).

If the \textit{group} clause is specified, the loops are merged in-order with do
statements sharing the same group in the current structured block (see example
\ref{loop-fusion2}).

The \textit{collapse} clause is used to specify how many tightly nested do
statements are associated with the \textbf{loop-fusion} construct (see example
\ref{loop-fusion4}). The argument to the \textit{collapse} clause is a constant
positive integer. If the \textit{collapse} clause is not specified, only the do
statement that follows immediately the directive is associated with the
\textbf{loop-fusion} construct.

\textbf{Options and details}
\begin{itemize}
\item \textit{group\_id}: An identifier that represents a group of do statement
in a structured block (see example \ref{loop-fusion2}).
\item \textit{n}: A constant positive integer.
\end{itemize}

\textbf{Behavior with other directives}\\
When the do statement to be merged are decorated with other directives, only the
directives on the first do statement of the merge group are kept in the
transformed code.

\textbf{Limitations}\\
All the do statement within a group must share the same iteration range. If the
\textit{collapse} clause is specified, the do statements must share the same
iteration range at the corresponding depth (see example \ref{loop-fusion4}).

\textbf{Code example}\\
\label{loop-fusion-1}
Example without clauses.

Original code
\begin{lstlisting}[language=Fortran]
DO k=1, iend
  !$claw loop-fusion
  DO i=1, iend
    ! loop #1 body here
  END DO

  !$claw loop-fusion
  DO i=1, iend
    ! loop #2 body here
  END DO
END DO
\end{lstlisting}

Transformed code
\begin{lstlisting}[language=Fortran]
DO k=1, iend
  ! CLAW transformation (loop-fusion same block group)
  DO i=1, iend
    ! loop #1 body here
    ! loop #2 body here
  END DO
END DO
\end{lstlisting}

More code examples in the appendix. Example with the \lstinline!group! clause
(see example \ref{loop-fusion2}). Example with others directives (see example
\ref{loop-fusion3}). Example with the \lstinline!collapse! clause (see example
\ref{loop-fusion4}).

% ------------------------------------------------------------------------------
%
% LOOP EXTRACT
%
% ------------------------------------------------------------------------------
\subsubsection{Loop extraction}
\lstset{emph={range_info,var,mapping,group_id,directives}, emphstyle=\itshape}
\begin{lstlisting}
!$claw loop-extract range(range_info) [[map(var[,var]...:mapping[,mapping]...) [map(var[,var]...:mapping[,mapping]...)] ...] [fusion [group(group_id)]] [parallel] [acc(directives)]
\end{lstlisting}

Loop extraction can be performed on a subroutine/function call. The do statement corresponding
to the defined iteration range is extracted from the subroutine/function and is wrapped
around the subroutine/function call. In the transformation, a copy of the subroutine/function
is created with the corresponding transformation (demotion) for the parameters.


\textbf{Options and details}
\begin{enumerate}
\item \textit{range}: Correspond to the induction variable of the do statement
  to be extracted. Notation \lstinline!i!.
\item \textit{map}: Define the mapping of variable that are demoted during the
  loop extraction. As seen in the example \ref{loop-extract1}, the two
  parameters (1 dimensional array) are mapped to a scalar with the induction
  variable \textit{i}.
  \begin{enumerate}
    \item The \textit{var} can be defined as two parts variable (e.g.
    \lstinline!a/a1!). The first part is the function call part and refers to
    the variable as it is defined in the function call. The second part is the
    function definition part and refers to the name of the variable to be mapped
    as it defined in the function declaration. If a \textit{var} is defined as
    a single part variable, the same name is used for both the function call
    and function definition part.
    \item The \textit{mapping} can be defined as two parts variable (e.g.
    \lstinline!i/i1!). The first part is the function call part and refers
    to the mapping variable as it is defined in the function call. The second
    part is the function definition part and refers to the name of the mapping
    variable as it defined in the function declaration. If a \textit{mapping}
    is defined as a single part mapping variable, the same name is used for
    both the function call and function definition part.
    \end{enumerate}
\item \textit{fusion}: Allow the extracted loop to be merged with other loops.
Options are identical with the \lstinline!loop-fusion! directive.
\item \textit{parallel}: Wrap the extracted loop in a parallel region.
\item \textit{acc}: Add the accelerator directives to the extracted loop.
\end{enumerate}

If the directive \textbf{loop-extract} is used for more than one call to the
same subroutine, the extraction can generate 1 to N dedicated subroutines.

\textbf{Behavior with other directives}\\
Accelerator directives are generated by the transformation. If the function call
was decorated with accelerator directives prior to the transformation, those
directives stay in place.

\textbf{Limitations}\\
The do statement to be extracted must be fully embraced in the
subroutine/function. This means that no statements must be oustide of the do
statement.

\textbf{Code example}\\
\label{loop-extract1}
Example without additional clauses.

Original code
\begin{lstlisting}[language=Fortran]
PROGRAM main
  !$claw loop-extract range(i) map(value1,value2:i)
  CALL xyz(value1, value2)
END PROGRAM main

SUBROUTINE xyz(value1, value2)
  REAL, INTENT (IN) :: value2(x:y), value2(x:y)

  DO i = istart, iend
    ! some computation with value1(i) and value2(i) here
  END DO
END SUBROUTINE xyz
\end{lstlisting}


Transformed code
\begin{lstlisting}[language=Fortran]
PROGRAM main
  !CLAW extracted loop
  DO i = istart, iend
    CALL xyz_claw(value1(i), value2(i))
  END DO
END PROGRAM main

SUBROUTINE xyz(value1, value2)
  REAL, INTENT (IN) :: value2(x:y), value2(x:y)

  DO i = istart, iend
    ! some computation with value1(i) and value2(i) here
  END DO
END SUBROUTINE xyz

!CLAW extracted loop new subroutine
SUBROUTINE xyz_claw(value1, value2)
  REAL, INTENT (IN) :: value1, value2
  ! some computation with value1 and value2 here
END SUBROUTINE xyz_claw
\end{lstlisting}

More code examples in the appendix. Example with the \lstinline!fusion! clause
(see example \ref{loop-extract2}). Example with a more complicated mapping
(see example \ref{loop-extract3}).

% ------------------------------------------------------------------------------
%
% LOOP HOISTING
%
% ------------------------------------------------------------------------------
\subsubsection{Loop hoisting}
\lstset{emph={induction_var,array_name,target_dimension,kept_dimension}, emphstyle=\itshape}
\begin{lstlisting}
!$claw loop-hoist(induction_var[[, induction_var] ...]) [reshape(array_name(target_dimension[,kept_dimensions]))] [interchange [(induction_var[[, induction_var] ...])]] [fusion [group(group_id)] [collapse(n)]] [cleanup[(acc|omp)]]

  ! structured block of code

!$claw end loop-hoist
\end{lstlisting}

The \textbf{loop-hoist} directive allow single or nested loops in a defined
structured block to be merged together and to hoist the beginning of those
nested loop just after the directive declaration. Loops with different
lower-bound indexes can also be merged with the addition of an \lstinline!IF!
statement. This feature works only when the lower-bound are integer constant.


\textbf{Options and details}
\begin{enumerate}
\item \textit{induction\_var}: List of induction variables of the do statements
to be hoisted.
\item \textit{interchange}: Allow the group of hoisted loops to be reordered
after the current transformation has been performed.
Options are identical with the \textbf{loop-interchange} directive.
\item \textit{reshape}: Reshape arrays to scalar or to array with fewer
dimensions. The original declaration is replaced with the corresponding demoted
declaration in the current block (function/module).
\item \textit{cleanup}: The cleanup clause will remove any OpenACC or OpenMP pragmas
already present in the hoit block. If OpenACC or OpenMP is not specified, both are
removed.

\begin{enumerate}
  \item \textit{array\_name}: Name of the array type to be reshaped.
  \item \textit{target\_dimension}: Number of dimension after reshaping. 0
  indicates that the type is reshapped to a scalar.
  \item \textit{kept\_dimension}: comma separated list of integer that indicates
  which dimensions are kept when the array is not totally demoted to scalar.
  Dimension index starts at 1.
\end{enumerate}
\item \textit{fusion}: Allow the extracted loop to be merged with other loops
after the current transformation has been performed.
Options are identical with the \lstinline!loop-fusion! directive.
\end{enumerate}

\textbf{Limitations}\\
If do statements in the hoist group need the addition of a \lstinline|IF|
statement, the lower-bound must be an integer constant.

\textbf{Code example}\\
\label{loop-hoist1}
Example without clauses.

Original code
\begin{lstlisting}[language=Fortran]
!$acc parallel loop gang vector collapse(2)
DO jt=1,jtend
  !$claw loop-hoist(j,i) interchange
  IF ( .TRUE. ) CYCLE
    ! outside loop statement
  END IF
  DO j=1,jend
    DO i=1,iend
      ! first nested loop body
    END DO
  END DO
  DO j=2,jend
    DO i=1,iend
      ! second nested loop body
    END DO
  END DO
  !$claw end loop-hoist
END DO
!$acc end parallel
\end{lstlisting}


Transformed code
\begin{lstlisting}[language=Fortran]
!$acc parallel loop gang vector collapse(2)
DO jt=1,jtend
  DO i=1,iend
    DO j=1,jend
      IF ( .TRUE. ) CYCLE
        ! outside loop statement
      END IF
      ! first nested loop body
      IF(j>1) THEN
        ! second nested loop body
      END IF
    END DO
  END DO
END DO
!$acc end parallel
\end{lstlisting}

% ------------------------------------------------------------------------------
%
% IF EXTRACT
%
% ------------------------------------------------------------------------------
\subsubsection{Conditional extraction from loop}
\begin{lstlisting}
!$claw if-extract
\end{lstlisting}

Conditional extraction from loop allows to extract a perfectly nested if
statement from a do statement.

\textbf{Options and details}\\
None.

\textbf{Behavior with other directives}\\
None.

\textbf{Limitations}\\
If the condition uses the induction variable, the transformation cannot be
performed.

\textbf{Code example}\\
\label{if-extract-1}
Example without clauses.

Original code
\begin{lstlisting}[language=Fortran]
!$claw if-extract
DO i = 1, 10
  IF (test) THEN
    PRINT *, 'Then body:', i
  ELSE
    PRINT *, 'Else Body', i
  END IF
END DO
\end{lstlisting}

Transformed code
\begin{lstlisting}[language=Fortran]
IF (test) THEN
  DO i = 1, 10
    PRINT *, 'Then body:', i
  END DO
ELSE
  DO i = 1, 10
    PRINT *, 'Else Body', i
  END DO
END IF
\end{lstlisting}


% ------------------------------------------------------------------------------
%
% LOOP FISSION
%
% ------------------------------------------------------------------------------
\subsubsection{Loop fission/splitting}
\begin{lstlisting}
!$claw loop-fission
\end{lstlisting}

Loop fission is used to split a do statement into multiple loops over the same 
iteration range with each taking only a part of the original loop's body. 
The goal is to break down a large loop body into smaller ones to achieve 
better utilization of locality of reference or also reduce pressure on 
the register usage.

\textbf{Options and details}
None.

\textbf{Behavior with other directives}\\
None.

\textbf{Code example}\\\label{loop-fission1}

Original code
\begin{lstlisting}[language=Fortran]
DO k = 1, kend

  !  Part 1 of the loop body 

  !$claw loop-fission

  !  Part 2 of the loop body 

END DO
\end{lstlisting}

Transformed code
\begin{lstlisting}[language=Fortran]
DO k = 1, kend

  !  Part 1 of the loop body 

END DO

DO k = 1, kend

  !  Part 2 of the loop body 

END DO
\end{lstlisting}
