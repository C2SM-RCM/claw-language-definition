\subsection{Utilities}

\subsubsection{Remove}
\begin{lstlisting}
!$claw remove

  ! code block
  
[!$claw end remove]
\end{lstlisting}

The \textbf{remove} directive allows the user to remove section of code
during the transformation process. This code section is lost in the 
transformed code.

\textbf{Options and details}\\
If the directive is directly followed by a structured block (\lstinline!IF! or \lstinline!DO!), the
end directive is not mandatory (see example \ref{remove1}). In any other cases, the end
directive is mandatory.

\textbf{Code example}
\label{remove1}

Original code
\begin{lstlisting}
DO k=1, kend
  DO i=1, iend
    ! loop #1 body here
  END DO

  !$claw remove
  IF (k > 1) THEN
    PRINT*, k
  END IF

  DO i=1, iend
    ! loop #2 body here
  END DO
END DO
\end{lstlisting}

Transformed code
\begin{lstlisting}
DO k=1, kend
  DO i=1, iend
    ! loop #1 body here
  END DO

  DO i=1, iend
    ! loop #2 body here
  END DO
END DO
\end{lstlisting}

More code examples in the appendix. Example with block remove (see example \ref{remove2}).


\subsubsection{Ignore}
\begin{lstlisting}
!$claw ignore

  ! code block
  
!$claw end ignore
\end{lstlisting}

The \lstinline!ignore! directive allows the user to ignore section of code
for the transformation process. This code section is retrieved in the 
transformed code. Any CLAW directives in an ignored code section is also
ignored.

\textbf{Code example}\\
\label{ignore1}
In the example below, the first remove block is ignored and the second is not.

Original code
\begin{lstlisting}
PROGRAM testignore

  !$claw ignore
  !$claw remove
  PRINT*,'These lines'
  PRINT*,'are ignored'
  PRINT*,'by the CLAW compiler'
  PRINT*,'but kept in the final transformed code'
  PRINT*,'with the remove directives.'
  !$claw end remove
  !$claw end ignore

  !$claw remove
  PRINT*,'These lines'
  PRINT*,'are not ignored.'
  !$claw end remove

END PROGRAM testignore
\end{lstlisting}

Transformed code
\begin{lstlisting}
PROGRAM testignore

  !$claw remove
  PRINT*,'These lines'
  PRINT*,'are ignored'
  PRINT*,'by the CLAW compiler'
  PRINT*,'but kept in the final transformed code'
  PRINT*,'with the remove directives.'
  !$claw end remove

END PROGRAM testignore
\end{lstlisting}

\subsubsection{Verbatim}
\begin{lstlisting}
!$claw verbatim IF (i > 5) THEN

  ! code block
  
!$claw verbatim END IF
\end{lstlisting}

The \lstinline!verbatim! directive allows the user to insert code after the
transformation process. This code is not analyzed by the compiler. 

\textbf{Code example}\\
\label{verbatim1}
In the example below, the first remove block is ignored and the second is not.

Original code
\begin{lstlisting}
PROGRAM testverbatim

  !$claw verbatim IF (.FALSE.) THEN
  PRINT*,'These lines'
  PRINT*,'are not printed'
  PRINT*,'if the the CLAW compiler has processed'
  PRINT*,'the file.'
  !$claw verbatim END IF

END PROGRAM testverbatim
\end{lstlisting}

Transformed code
\begin{lstlisting}
PROGRAM testverbatim

IF (.FALSE.) THEN
PRINT*,'These lines'
PRINT*,'are not printed'
PRINT*,'if the the CLAW compiler has processed'
PRINT*,'the file.'
END IF

END PROGRAM testverbatim
\end{lstlisting}
